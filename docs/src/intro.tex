\section{Introduction}\label{sec:intro}

The PSBLAS-EXT  library contains a set of extensions to the base
library. The extensions provide additional storage formats beyond the
 ones already contained in the base library, as well as interfaces to
 two external libraries:
\begin{itemize}
\item SPGPU \url{https://code.google.com/p/spgpu/}, for computations on
  NVIDIA GPUs;
\item LIBRSB \url{http://sourceforge.net/projects/librsb/}, for
  computations on multicore parallel machines. 
\end{itemize}
The infrastructure laid out in the base library to allow for these
extensions is detailed in the references~\cite{CaFiRo:14,Sparse03}. 


\subsection{Application structure}
\label{sec:appstruct}
A sample application using the PSBLAS extensions will contain the
following steps:
\begin{itemize}
\item \verb|USE| the appropriat modules (\verb|psb_ext_mod|,
  \verb|psb_gpu_mod|);
\item Declare a \emph{mold} variable of the necessary type
  (e.g. \verb|psb_d_ell_sparse_mat|, \verb|psb_d_hlg_sparse_mat|,
  \verb|psb_d_vect_gpu|);
\item Pass the mold variable to the base library interface where
  needed to ensure the appropriate dynamic type.
\end{itemize}
Suppose you want to use the GPU-enabled ELLPACK data structure; you
would use a piece of code like this (and don't forget, you need
GPU-side vectors along with the matrices):
\lstset{language=Fortran}
\begin{lstlisting}
program my_gpu_test
  use psb_base_mod
  use psb_util_mod 
  use psb_ext_mod
  use psb_gpu_mod
  type(psb_dspmat_type) :: a, agpu
  type(psb_d_vect_type) :: x, xg, bg

  real(psb_dpk_), allocatable :: xtmp(:)
  type(psb_d_vect_gpu)       :: vmold
  type(psb_d_elg_sparse_mat) :: aelg

  ......  

  ! My own home-grown matrix generator
  call gen_matrix(ictxt,idim,desc_a,a,x,info)
  
  call a%cscnv(agpu,info,mold=aelg)
  xtmp = x%get_vect() 
  call xg%bld(xtmp,mold=vmold)
  call bg%bld(size(xtmp),mold=vmold)
  
  ! Do sparse MV
  call psb_spmm(done,agpu,xg,dzero,bg,desc_a,info)
\end{lstlisting}
A full example of this strategy can be seen in the
\verb|test/ext/kernel| subdirectory, where we provide a sample program
to test the speed of the sparse matrix-vector product with the various
data structures included in the library. 





%%% Local Variables: 
%%% mode: latex
%%% TeX-master: "userguide"
%%% End: 
