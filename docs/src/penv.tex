
\section{GPU Environment Routines}
\label{sec:gpuenv}

\subsection*{psb\_gpu\_init --- Initializes PSBLAS-GPU
  environment}
\addcontentsline{toc}{subsection}{psb\_gpu\_init}

\begin{verbatim}
call psb_gpu_init(icontxt [, device])
\end{verbatim}

This subroutine initializes the PSBLAS-GPU  environment. 
\begin{description}
\item[Type:] Synchronous.
\item[\bf  On Entry ]
\item[device] ID of GPU device to attach to.\\
Scope: {\bf local}.\\
Type: {\bf optional}.\\
Intent: {\bf in}.\\
Specified as: an integer value. \
Default: use \verb|mod(iam,ngpu)| where \verb|iam| is the calling
process index and \verb|ngpu| is the total number of GPU devices
available on the current node. 
\end{description}


{\par\noindent\large\bfseries Notes}
\begin{enumerate}
\item A call to this routine must precede any other PSBLAS-GPU call. 
\end{enumerate}

\subsection*{psb\_gpu\_exit --- Exit from  PSBLAS-GPU
  environment}
\addcontentsline{toc}{subsection}{psb\_gpu\_exit}

\begin{verbatim}
call psb_gpu_exit(icontxt)
\end{verbatim}

This subroutine exits from the  PSBLAS parallel virtual  machine.
\begin{description}
\item[Type:] Synchronous.
\item[\bf  On Entry ]
\item[icontxt] the communication context identifying the virtual
  parallel machine.\\
Scope: {\bf global}.\\
Type: {\bf required}.\\
Intent: {\bf in}.\\
Specified as: an integer variable.
\end{description}




\subsection*{psb\_gpu\_DeviceSync ---  Synchronize GPU device}
\addcontentsline{toc}{subsection}{psb\_gpu\_DeviceSync}

\begin{verbatim}
call psb_gpu_DeviceSync()
\end{verbatim}

This subroutine ensures that all previosly invoked kernels, i.e. all
invocation of GPU-side code, have completed.


\subsection*{psb\_gpu\_getDeviceCount }
\addcontentsline{toc}{subsection}{psb\_gpu\_getDeviceCount}

\begin{verbatim}
ngpus =  psb_gpu_getDeviceCount()
\end{verbatim}

Get number of devices available on current computing node. 

\subsection*{psb\_gpu\_getDevice }
\addcontentsline{toc}{subsection}{psb\_gpu\_getDevice}

\begin{verbatim}
ngpus =  psb_gpu_getDevice()
\end{verbatim}

Get  device in use by current process. 

\subsection*{psb\_gpu\_setDevice }
\addcontentsline{toc}{subsection}{psb\_gpu\_setDevice}

\begin{verbatim}
info = psb_gpu_setDevice(dev)
\end{verbatim}

Set  device to be used  by current process. 

\subsection*{psb\_gpu\_DeviceHasUVA }
\addcontentsline{toc}{subsection}{psb\_gpu\_DeviceHasUVA}

\begin{verbatim}
hasUva = psb_gpu_DeviceHasUVA()
\end{verbatim}

Returns true if device currently in use supports UVA (Unified Virtual Addressing).


